\documentclass[../main.tex]{subfiles}
\begin{document}

\section{Computational Neuroscience: a brief introduction}
In the last years, the field of Computational Neuroscience has rapidly evolved into a crucial intersection between neuroscience and computer science.
Its primary objective is to reveal the fundamental principles that govern neural dynamics and interactions.
This investigation occurs both at the local level within specific brain regions and neuronal clusters as well as on a global scale.
The global perspective involves understanding communication across distant brain regions, operating across a broad spectrum of spatial and temporal scales.

At the heart of computational neuroscience lies the mathematical modelling of neural systems, with a focal point on neurons and synapses.
Neurons, serving as the elemental units of computation in the brain, are responsible for processing and transmitting information through intricate synapses.
These neurons exhibit diverse behaviors, ranging from simple firing patterns to complex oscillations, and can be classified based on their structural and functional properties.
Synapses, acting as crucial bridges between neurons, facilitate the transmission of electrical or chemical signals.
These signals, with modifiable strengths, form the foundation of synaptic plasticity, a crucial mechanism for learning and memory.

Computational network models play a crucial role in capturing the dynamic interplay between neurons and synapses, replicating intricate neural dynamics observed in the brain.
These models provide a comprehensive framework for exploring the processing, representation, and propagation of information within neural networks.
Their integration with experimental data has significantly improved our understanding of how neural systems compute and operate dynamically.
Computational models, acting as powerful tools, enable hypothesis testing, predictions, and a deeper understanding of brain functions.

Computational neuroscience has been crucial to advance technologies and techniques for studying the brain.
Progress in this field has greatly contributed to the interpretation and analysis of neuroimaging data, e.g., functional magnetic resonance imaging (fMRI) and electroencephalography (EEG).
These techniques offer insights into brain activity across different spatial and temporal scales, and computational models aid in unraveling complex patterns and relationships within these datasets.

% The continuous evolution of computational neuroscience not only enriches our understanding of the brain but also contributes to the development of innovative tools and techniques for studying and interpreting neural activity.
Through ongoing research and interdisciplinary collaborations, this field continues to push the boundaries of knowledge, paving the way for exciting discoveries in neuroscience and beyond.
\end{document}

In this chapter, we will explore the key concepts and techniques for modeling neural dynamics.
We will begin by delving into the fundamentals of neuron and synapse models, which serve as the foundational building blocks of neural networks. 
We will examine various types of neuron models, encompassing biophysical models, integrate-and-fire models, and conductance-based models. 
By understanding these models, we can capture the diverse behaviors exhibited by neurons, ranging from intricate biophysical processes to simplified spiking dynamics.
Additionally, we will explore different types of synapse models which play a crucial role in network connectivity and information transmission.

Moving forward, we will delve into network models, which enable us to simulate the interactions among neurons within a given system.
We will investigate different types of network models, such as feed-forward and recurrent networks, each offering unique characteristics and capabilities for studying neural processes. Furthermore, we will explore the influence of network topology on the emergent behaviors of neural networks, highlighting the significance of network architecture in shaping network dynamics. Through the use of simulations, we can gain insights into the collective behavior and information processing capabilities of neural networks, unraveling the underlying mechanisms of neural processes (\textbf{esta parte la tengo dudosa})

In summary, this chapter will provide a comprehensive overview of the key concepts and techniques involved in modeling artificial neural networks. 
By exploring neuron models, synapse models, and network models, we aim to capture the intricacies of neural dynamics and understand the fundamental principles governing information processing in the brain. 
Furthermore, we will examine the current state of the field, shedding light on the advancements made in neural network modeling and the exciting possibilities that lie ahead (\textbf{esta parte también}).
