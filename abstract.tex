\documentclass[../main.tex]{subfiles}

\begin{document}
\begin{center}
    \textbf{Abstract}
\end{center}

In this doctoral thesis, we dive into the field of computational neuroscience, exploring various aspects of neural network modeling.
Our work unfolds in three interconnected parts, each contributing to a comprehensive understanding of information processing in the brain.
In the first part, we investigate the complexities of information transmission within cortical circuits.
The focus revolves around unraveling the essential conditions for effective communication between interconnected cortical populations mediated by a potential third element, the thalamus.
We find two operation modes dynamically switchable by modulating the strength of the direct cortico-cortical connection.
One of them supports robust thalamocortical transmission and corticothalamic feedback in weak-connection scenarios.
The other, with stronger cortico-cortical connections, fosters the coexistence of canonical thalamo-cortical transmission and a cortico-thalamo-cortical association loop, contributing to cortical computations.

In the second part, we explore the modeling of data-driven whole-brain networks with spiking neurons.
Here, we aim to harness the power of computational models to investigate potential therapies based on transcranial current stimulation (tACS) patients facing various conditions.
We emphasize the use of spiking models for the design of such networks, where traditionally simpler and more abstract models such as neural mass models have been considered.
We have developed a methodology to optimize these networks, subsequently applying tACS to increase the alpha band power on personalized networks based on patient data.
This experiment represents a pivotal step towards a proposed clinical treatment.

Finally, we model the hippocampal region using multicompartmental neuron models, capturing the intricate dynamics of the CA3 and CA1 subfields.
In this circuit, we study the interaction between theta and gamma rhythms, crucial in facilitating cognitive processes such as memory formation and spatial navigation.
With a simplified hippocampal model, we can reproduce observed phenomena, such as the long delay between CA3 and CA1 pyramidal cells and the role of inhibition in the enhacement of theta-gamma cross-frequency coupling.

In conclusion, through the modeling of neural networks, we conduct our study of neuronal phenomena at different scales.
From the study of small circuits in the initial phase to the exploration of the complexities of a complete brain network in the second, we culminate in the detailed modeling of a specific brain region in the third.
This not only enriches the understanding of neuronal dynamics but also emphasizes the crucial role of scale in modeling, aiding in unraveling the complexities of the brain across multiple organizational levels.
\end{document}