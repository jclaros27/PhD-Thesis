\documentclass[../main.tex]{subfiles}
\graphicspath{{\subfix{../figs/}}}
\begin{document}

%*******************************************************
% Acknowledgments
%*******************************************************
\pagenumbering{gobble}
\pdfbookmark[1]{Acknowledgments}{acknowledgments}
\begingroup
\let\clearpage\relax
\let\cleardoublepage\relax
\let\cleardoublepage\relax
\chapter*{Agradecimientos}
\thispagestyle{empty}
Tras una temporada en el Reino Unido con mi buen amigo Jorge, tomé la decisión de cambiar de rumbo.
Volví a España, pero esta vez tocó instalarse en Palma de Mallorca. Tras superar el proceso de selección, pude comenzar el programa de doctorado gracias a uno de los contratos de María Maeztu, bajo la supervisión de Claudio Mirasso y de Santiago Canals.
En ese momento, me adentré en el mundo de la Neurociencia Computacional.
Un camino apasionante en el que he adquirido un conocimiento que años atrás jamás podría haber imaginado.
Aunque claro, no siempre ha sido fácil.
Ha habido períodos desmotivadores en los que había que sacar optimismo fuera como fuera.
Pero finalmente, aquí estamos, finalizando la redacción del trabajo de todos estos años.

Me gustaría comenzar agradeciendo la buena acogida que recibí una vez me instalé en Palma.
Al igual que yo, fueron contratados otros doctorandos, a quienes más tarde denominé (o denominamos) la primera generación María Maeztu.
Sin ellos, y junto a Adri, explorar todo lo que ofrecía Mallorca no habría sido lo mismo.
Especial mención a Marco, a Mattia y a Thomas por el tiempo que hemos vivido juntos.
Hemos disfrutado de mucho cine, gracias al espíritu cultural de Marco, de divertidas partidas de Catán (algo más que un juego), al que a día de hoy seguimos jugando online, y de nuestras incontables discusiones políticas.
Ya fuera para politizar a uno o para destruir el pensamiento post-68 del otro. Aprendimos mucho, jugamos mucho, y sufrimos un confinamiento juntos, aunque aquí Thomas se pudo librar.

Más tarde, el instituto seguía creciendo y nuevos doctorandos se incorporaban.
Fue aquí cuando conocí a los canarios Javi y Medi, quienes no tardaron en convertirse en muy buenos amigos.
Resultaron ser un punto de inflexión en mi vida.
Fueron el punto de partida para lo que hoy es una parte esencial de mi vida: el entrenamiento (las inversiones llegaron más tarde, pero llegaron).
Creo que son bastante conscientes de ello, pero nunca está de más recordarlo una vez más.
Al tiempo, y unidos al último elemento con quien logramos la cuadratura del círculo, Jojito, nos fuimos a vivir juntos.
A partir de ese momento nos denominaron "la Secta".
Sin duda, siempre echaré de menos el clima que creamos en ese piso.

Me gustaría mencionar a toda la gente del Instituto de Neurociencias de Alicante y a la gente del Laboratorio de Poirazi con los que he tenido el placer de conocer.
Del primero, destacar a Cristian, que rápidamente congeniamos y se convirtió en un buen amigo con el que era muy difícil no empezar a reírse.
Del segundo, destaco a Spyros, quien me ayudó enormemente en mi estancia en Creta, y sin sus consejos y propuestas, se me habría hecho muy difícil llegar hasta donde estoy.

Agradezco mucho la compañía de Dimitris, con quien tuve la experiencia de enseñar todos mis conocimientos y todo el trabajo que había realizado.
Al poco tiempo se convirtió en una gran ayuda y apoyo en mis últimos pasos como doctorando. Sin todos aquellos paseos en los que te desarrollaba todas mis ideas y todos aquellos trabajitos que te mandaba, no habría sido lo mismo llegar hasta aquí.
Ojalá hubieras llegado antes, pero llegaste, y eso es lo importante.

Por supuesto, mencionar a todos aquellos que, desde la distancia, siempre han estado conmigo y siempre han creído en mí: Jose, Daniele, María, Joseda, Jaime, Tomás y Comazzi.

Por último, y no menos importante, estaré eternamente agradecido a Victoria.
Hemos recorrido nuestros caminos juntos, apoyándonos mutuamente en los momentos difíciles y celebrando los buenos, y todo ello, a pesar de la distancia.
Afortunadamente, hemos tenido la posibilidad de convivir varias veces juntos, lo cual ha sido maravilloso. Hemos creado innumerables recuerdos de todas nuestras aventuras y vivencias, y cada vez somos más creativos con el idioma que inventamos.
Siempre sabes cómo hacerme reír, por muy mal que haya podido estar, siempre has sabido cómo sacar la carcajada más fuerte que soy capaz de producir.
Espero que nuestros caminos confluyan para estar más cerca el uno del otro.

\endgroup
\cleardoublepage
\end{document}
Han sido algo más de cuatro años el tiempo que me ha llevado realizar mi tesis doctoral.
Tras una termporada en Reino Unido con mi buen amigo Jorge, tomé la decisión de cambiar de rumbo.
Volví a España, pero esta vez tocó instalarse en Palma de Mallorca.
Tras superar el proceso de selección, pude comenzar el programa de doctorado gracias a uno de los contratos de María Maeztu, bajo la supervisión de Claudio Mirasso y de Santiago Canals.
En ese momento, me adentré en el mundo de la Neurociencia Computacional.
Un camino apasionante que en el que he adquirido un conocimiento que años atrás jamás podría haber imaginado.
Aunque claro, no siempre ha sido fácil.
Ha habido períodos desmotivadores en los que había que sacar optimismo fuera como fuera.
Pero finalmente, aquí estamos, finalizando la redacción del trabajo de todos estos años.

Me gustaría empezar agradeciendo la buena acogida que recibí una vez me instalé en Palma.
Al igual que yo, fueron contratados otros doctorandos, los que más tarde denominé (o denominamos) la primera generación María Maeztu.
Sin ellos, explorar la belleza de Mallorca no habría sido lo mismo.
Especial mención a Marco, a Mattia y a Thomas, por el tiempo en el que hemos vivido juntos.
Hemos disfrutado de mucho cine, gracias al cultureta de Marco, de divertidas de partidas de Catán (algo más que un juego), al que a día de hoy seguimos jugando online, y de nuestras incontables discusiones políticas. Ya fuera para politizar a uno o para destruir el pensamiento post-68 del otro.
Aprendimos mucho, jugamos mucho, y sufrimos un confinamiento juntos, aunque aquí Thomas se pudo librar.

Más tarde, el instituto seguía creciendo y nuevos doctorandos se incorporaban.
Fue aquí cuando conocí a los canarios Javi y Medi, quienes no tardaron en convertirse en muy buenos amigos.
Resultaron ser un punto de inflexión en mi vida.
Fueron el punto de partida para lo que hoy es una parte esencial de mi vida: el entrenamiento (las inversiones llegaron más tarde, pero llegaron).
Creo que son bastante consicentes de ello, pero nunca está demás recordarlo una vez más.
Al tiempo, y unidos al último elemento con quien logramos la cuadratura del círculo, Jojito, nos fuimos a vivir juntos.
A partir de ese momento nos denominaron "la Secta".
Sin duda, siempre echaré de menos el clima que creamos ese piso.

Me gustaría mencionar a toda la gente del Instituto de Neurociencias de Alicante y a la gente del Laboratorio de Poirazi con los que he tenido el placer de conocer.
Del primero, destacar a Cristian, que rápidamente congeniamos se convirtió en un buen amigo con el que era muy difícil no empezar a reírse.
Del segundo, destaco a Spyros, quien me ayudó enormemente en mi estancia en Creta, y sin sus consejos y propuestas, se me habría hecho muy difícil llegar hasta donde estoy.

Agradezco mucho la compañía de Dimitris, con quien tuve la experiencia de enseñar todos mis conocimientos y todo el trabajo que había realizado.
Al poco tiempo se convirtió en una gran ayuda y apoyo en mis últimos pasos como doctorando.
Sin todos aquellos paseos en los que te desarrollaba todas mis ideas y todos aquellos trabajitos que te mandaba, no habría sido lo mismo llegar hasta aquí.
Ojalá hubieras llegado antes, pero llegaste, y eso es lo importante.

Por supuesto, mencionar a todos aquellos, que desde la distancia siempre han estado conmigo y siempre han creído en mí: Jose, Daniele, María, Joseda, Jaime, Tomás y Comazzi.

Por último, y no menos importante, estaré eternamente agradecido a Victoria.
Hemos recorrido nuestros caminos juntos, apoyándonos mutuamente en los momentos dificles y celebrando los buenos, y todo ello, a pesar de la distancia.
Sin embargo, hemos tenido la posibilidad de convivir varias veces juntos, lo cual ha sido maravilloso.
Hemos creado innumerables recuerdos de todas nuestras aventuras y vivencias, y cada vez somos más creativos con el idioma que nos inventamos.
Siempre sabes como hacerme reír, por muy  muy mal que haya podido estar, siempre has sabido como sacar la carcajada más sonora que soy capaz de producir.
Espero que nuestros caminos confluyan a estar más cerca el uno del otro.