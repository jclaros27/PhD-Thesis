\documentclass[../main.tex]{subfiles}
\begin{document}

%*******************************************************
% Abstract
%*******************************************************
%\renewcommand{\abstractname}{Abstract}
\pagenumbering{gobble}
\pdfbookmark[1]{Abstract}{Abstract}
% \addcontentsline{toc}{chapter}{\tocEntry{Abstract}}
\begingroup
\let\cleardoublepage\relax
\let\cleardoublepage\relax

\chapter*{Abstract}
% \thispagestyle{empty} % <=======
\section*{English}
In this doctoral thesis, we dive into the field of Computational Neuroscience, exploring various aspects of neural network modelling.
Our work unfolds in three interconnected parts, each contributing to a comprehensive understanding of information processing in neural networks.
In the first part, we investigate the complexities of information transmission within cortical circuits.
The focus revolves around unraveling the essential conditions for effective communication between interconnected cortical populations mediated by a potential third element, the thalamus.
We find two operation modes dynamically switchable by modulating the strength of the direct cortico-cortical connection.
% One of them supports robust thalamocortical transmission and corticothalamic feedback in weak-connection scenarios.
% The other, with stronger cortico-cortical connections, fosters the coexistence of canonical thalamo-cortical transmission and a cortico-thalamo-cortical association loop, contributing to cortical computations.

In the second part, we explore the modelling of data-driven mesoscale brain networks with spiking neurons.
Here, we aim to harness the power of computational models to investigate potential therapies based on transcranial current stimulation (tACS) for patients facing various conditions.
We emphasize the use of spiking neuron models for the design of such networks, where traditionally simpler and more abstract models such as neural mass models have been considered.
We have developed a methodology to optimize these networks, subsequently applying tACS to increase the alpha band power on personalized networks based on patient data.
This experiment represents a pivotal step towards a proposed clinical treatment.

Finally, we model the hippocampal region using multicompartmental neuron models, capturing the intricate dynamics of the CA3 and CA1 subfields.
In this circuit, we study the interaction between theta and gamma rhythms, crucial in facilitating cognitive processes such as memory formation and spatial navigation.
With a simplified hippocampal model, we can reproduce observed phenomena, such as the long delay between CA3 and CA1 pyramidal cells and the role of inhibition in the enhancement of theta-gamma cross-frequency coupling.

Through the modelling of neural networks, we conduct our study of neuronal phenomena at different scales.
From the study of small population circuits in the initial phase to the exploration of the complexities of mesoscale brain networks in the second, we culminate in the detailed modelling of a specific brain region in the third.
This not only enriches the understanding of neuronal dynamics but also emphasizes the crucial role of scale in modelling, aiding in unravelling the complexities of the brain across multiple organizational levels.
% \thispagestyle{empty}%<=======

% \clearpage
\section*{Català}
\begin{otherlanguage}{catalan}
% \pdfbookmark[1]{Resum}{Resum}
% \chapter*{Resum}
% \thispagestyle{empty}%<=======
En aquesta tesi doctoral, en el marc de la Neurociència Computacional, investiguem diversos aspectes de la modelització de xarxes neuronals.
Aquest treball es desenvolupa en tres parts interconnectades, cadascuna contribuint a una comprensió integral del processament de la informació al xarxes neuronals.

En la primera part, investiguem les complexitats de la transmissió d'informació en circuits corticals. El nostre enfocament pretén desxifrar les condicions essencials necessàries per aconseguir una comunicació efectiva entre poblacions corticals interconnectades mitjançant un possible tercer element, el tàlem.
Trobem dos modes d'operació dinàmicament intercanviables en funció de la intensitat de les connexions cortico-corticals.

En la segona part, explorem la modelització de \textit{data-drive mesoscale brain networks} amb \textit{spiking neuron models}.
Aquí, el nostre objectiu és aprofitar la capacitat dels models computacionals per investigar potencials teràpies basades en l'estimulació transcraneial de corrent alterna (tACS) en pacients que enfronten diverses condicions.
Posem l'èmfasi en l'ús de \textit{spiking neuron models} per al disseny d'aquestes xarxes, on tradicionalment s'han considerat models més simples i abstractes com els \textit{neural mass models}.
Desenvolupem una metodologia per optimitzar els paràmetres de xarxes personalitzades d'aquest tipus, i apliquem posteriorment tACS per augmentar la potència en la banda alfa.
Aquest experiment representa un pas crucial cap al tractament clínic proposat. 

Finalment, modelem la regió de l'hipocamp utilitzant models de neurones multicompartimentals, on capturem les intrincades dinàmiques de les regions CA3 i CA1.
En aquest circuit, estudiem la interacció entre els ritmes theta i gamma, crucials per facilitar processos cognitius com la formació de memòria i la navegació espacial.
Amb un model simplificat de l'hipocamp, som capaços de reproduir fenòmens observats, com el llarg retard entre les neurones piramidals CA3 i CA1 i el paper de la inhibició en l'augment de l'acoblament entre les bandes de freqüència gamma i theta.

A través de la modelització de xarxes neuronals, portem a terme un estudi de diferents fenòmens neuronals en diferents escales. Des de l ‘estudi de circuits de petites poblacions en la fase inicial fins a l'exploració de les complexitats d'una xarxa cerebral de tot el cervell en la segona, culminem en la modelització detallada d'una regió cerebral específica en la tercera.
Això no només enriqueix la comprensió de les dinàmiques neuronals involucrades, sinó que també posa de manifest el paper crucial de l'escala en la modelització, ajudant a desxifrar les complexitats del cervell en múltiples nivells d'organització.
\end{otherlanguage}

\section*{Castellano}
\begin{otherlanguage}{spanish}
% \pdfbookmark[1]{Resumen}{Resumen}
% \chapter*{Resumen}
% \thispagestyle{empty}%<=======
En esta tesis doctoral, en el marco de la Neurociencia Computacional, investigamos diversos aspectos del modelado de redes neuronales.
Este trabajo se desarrolla en tres partes interconectadas, cada una contribuyendo a una comprensión integral del procesamiento de información en redes neuronales.
En la primera parte, investigamos las complejidades de la transmisión de información en circuitos corticales.
Nuestro enfoque pretende desentrañar las condiciones esenciales necesarias para conseguir una comunicación efectiva entre poblaciones corticales interconectadas mediadas por un posible tercer elemento, el tálamo.
Encontramos dos modos de operación dinámicamente intercambiantes en función de la intensidad de las conexiones cortico-corticales.
% Uno de ellos respalda una sólida transmisión tálamo-cortical y retroalimentación corticotálamica en escenarios de conexión débil.
% El otro, con conexiones cortico-corticales más fuertes, fomenta la coexistencia de la transmisión tálamo-cortical canónica y un bucle de asociación corticotálamico, contribuyendo a las computaciones corticales.

En la segunda parte, exploramos el modelado de \textit{data-driven mesoscale brain networks} con \textit{spiking neuron models}.
Aquí, nuestro objetivo es aprovechar la capacidad de los modelos computacionales para investigar potenciales terapias basadas en la estimulación transcraneal de corriente alterna (tACS) en pacientes que enfrentan diversas condiciones.
Hacemos hincapié en el uso de \textit{spiking neuron models} para el diseño de tales redes, donde tradicionalmente se han considerado modelos más simples y abstractos como los \textit{neural mass models}.
Desarrollamos una metodología para optimizar los parámetros de redes personalizadas de este tipo, y aplicamosposteriormente tACS para aumentar la potencia en la banda alfa.
Este experimento representa un paso crucial hacia el tratamiento clínico propuesto.

Finalmente, modelamos la región del hipocampo utilizando modelos de neuronas multicompartimentales, donde capturamos las intrincadas dinámicas de las regiones CA3 y CA1.
En este circuito, estudiamos la interacción entre los ritmos theta y gamma, cruciales para facilitar procesos cognitivos como la formación de memoria y la navegación espacial.
Con un modelo simplificado del hipocampo, somos capaces de reproducir fenómenos observados, como el largo retraso entre las neuronas piramidales CA3 y CA1 y el papel de la inhibición en el aumento del acoplamiento entre las bandas de frecuencia gamma y theta.

A través del modelado de redes neuronales, llevamos a cabo un estudio de diferentes fenómenos neuronales en diferentes escalas.
Desde el estudio de circuitos de pequeñas poblaciones en la fase inicial hasta la exploración de las complejidades de una red cerebral de todo el cerebro en la segunda, culminamos en el modelado detallado de una región cerebral específica en la tercera.
Esto no solo enriquece la comprensión de las dinámicas neuronales involucradas, sino que también enfatiza el papel crucial de la escala en el modelado, ayudando a desentrañar las complejidades del cerebro en múltiples niveles de organización.
\end{otherlanguage}
% \clearpage

\endgroup
\vfill
\end{document}